\paragraph{Overview}
Optical heart rate monitors, the kind of which are standard in modern wearables, work by measuring the amount of light emitted by the surface of the skin 
as a proxy for understanding the blood flow beneath.

Given this signal, a heart rate can be extracted by analysing the predominant frequencies in this signal. In principle, 
this notion, can be extended to traditional cameras such as those present in smartphones, which also work by effectively measuring light intensity.
This project is concerned with the implementation of such a system which can infer the heart rate of a user from a video of that user, recorded by a simple camera such as those
present in a smartphone or similar device.

\paragraph{Motivation}
In general, sensing technology has progressed in recent times from high fidelity and low availability to low fidelity and high availability.
Heart rate sensing used to be the remit of expensive electrocardiogram devices in hospitals, but is now commonplace in relatively low cost wearable devices.
This trend allows increasing number of users to gain access to their own health data albeit at the cost of fidelity.
Smartphones are, I argue, in fact, of greater availability than wearables and, thereby, the development of heart rate sensing capabilities for 
smartphones represents a step of further progress within this wider trend. 
In fact, I will later argue that smartphones, with their increased computational power 
could be of greater fidelity than wearables for heart rate sensing, in which case, they represent a much more significant 
leap forward than might be anticipated.
% \paragraph{Literature}
%  This project, as a result, evaluates the value of 
% \section{Related work}
%talk about general trend in health sensing from high fidelity to low fidelity but high availaability
%smartphones are argubaly more available and so it's an extension of that trend