This project, by nature, combines a variety of disciplines and technologies, including, but not limited to, computer vision, sensing and signal processing. 
Before being able to proceed with the project, understanding the required topics in each of these respective fields was critical. Several relevant topics are outlined 
and described in relation to the project.

\section{Sensing and photoplethysmography (PPG)}
Before attempting to develop a system which can estimate the heart rate of a user without physical contact, it is useful to understand how traditional heart rate sensors work.
This understanding guides

\section{Remote photoplethysmography (rPPG)}
\subsection{Existing literature}
\subsection{}

\section{Relevant computer vision techniques}
\subsection{Face detection}
\subsection{Optical flow}
\subsection{Harris corner detection}
% \section{Conditional Random Fields}
% \section{Semantic Segmentation}
\section{Approaches to signal processing}
\subsection{Independent Component Analysis}
\subsection{Fourier transformation}
% \section{Halide}
% \section{RenderScript}
% \section{Differentiable Programming}


\section{Languages and tooling}
\subsection{Languages}
\paragraph{Python}
\paragraph{Kotlin}

\subsection{Libraries}
\paragraph{OpenCV}
\paragraph{Mobile Vision API}

\subsection{Android}

\section{Starting Point}

\section{Requirements analysis}


\section{Professional practice}
% planning 
% prototyping leverages speed of development in python
% android implementation completes the movement towards higher availability