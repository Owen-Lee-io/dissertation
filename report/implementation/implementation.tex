\section{Overview}
GIVE STRUCTURE OF PROGRAM AND EXPLAIN WHY MOST WORK IS DEDICATED TO REGION
SELECTION AND HEART RATE ISOLATION SPECIFICALLY

THAT IS FACE DETECTION IS MOSTLY A SOLVED PROBLEM
BACKGROUND ON HOW DESIGN IS TO MIMIC A PPG SENSOR

%\section{Face detection}
%\subsection{Repeated face detection}
%\subsection{Point tracking}
%As opposed to 

\section{Region of interest selection}
DEFINE THE BASELINE: PRIMITIVEROI
\subsection{Facial landmarks}
\subsection{Skin detection}
Face detection systems, typically, return a bounding box, within which it is believed
a face is present. However, naturally, the box will also contain pixels from the background of 
the image since faces are not, in general, perfectly rectangular.
These background pixels will not contain any information as to the underlying heart rate of the user.
As a result, considering the entire bounding box will add unecessary noise to the resulting signal.
The ideal algorithm would only consider skin pixels, however, robust, pose-invariant skin detection is an unsolved problem.
On the other hand, considering too few pixels could increase the effect of specular reflection.

\subsection{Clustering approaches}
\subsubsection{K-Means clustering}
TALK ABOUT COMPLEXITY OF K-MEANS AND WHY USING IT REPEATEDLY DOESN'T WORK THAT WELL
%\subsubsection{Hierarchical clustering}
\subsubsection{Markov clustering}

%\subsection{}
%\subsection{Flood filling}
\subsection{Colour Space Filtering}
EXPLAIN PERCEPTUAL UNIFORMITY 
SHOW YCbCr REPRESENTATION OF SKIN PIXELS
%\subsection{Conditional Random Fields}

% \section{Ensemble Methods}

\section{Heart rate isolation}
EXPLAIN WHY IT'S NOT JUST THE FOURIER POWER SPECTRUM
IN VIDEOS WITH MOVEMENT WE EXPECT HEART RATE TO BE A SERIES OF PEAKS TOGETHER RATHER THAN A SINGLE PEAK
THAT MIGHT BE DUE TO LIGHTING ETC
IMPACT OF LIGHTING CONDITIONS

\subsection{Blind-source separation}
ASSUMPTION OF CAMERA FEED BEING A MIXTURE OF INDEPENDENT SOURCES INCLUDING HEART RATE

SELECTION OF RESULTING COMPONENT => HIGHEST PEAK

%SHOW IDEALISED HR AND WHY AUTOCORRELATION HELPS US PICK IT OUT


\subsection{Respiration rejection}
%\subsection{}
