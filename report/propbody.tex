
% Draft #1 (final?)

\vfil

\centerline{\Large Computer Science Tripos - Part II - Project Proposal}
\vspace{0.1in}
\centerline{\huge \textbf{Non-contact heart rate estimation from video}}
\vspace{0.1in}
\centerline{\large Yousuf Mohamed-Ahmed, Gonville and Caius College}
\vspace{0.4in}
\leftline{\large \textbf{Originator}: Dr Robert Harle}
\vspace{0.1in}
\leftline{\large \textbf{Supervisor}: Dr Robert Harle}
\vspace{0.1in}
\leftline{\large \textbf{Directors of Studies}: Timothy Jones, Graham Titmus and Peter Robinson}
\vspace{0.1in}
\leftline{\large \textbf{Overseers}: Rafal Mantiuk and Andrew Pitts}
\vfil

% \title{Computer Science Tripos - Part II - Project Proposal \\
% \textbf{Non-contact heart rate estimation from video}}
% \author{Yousuf Mohamed-Ahmed \vspace{3ex}}


% \begin{document}
% \date{\vspace{-5ex}}
% \maketitle
%  \noindent
%  \textbf{Project Originator:} Robert Harle \\
% \textbf{Project Supervisor:} Robert Harle \\
% \textbf{Director of Studies:} Timothy Jones and Graham Titmus \\
% \textbf{Overseers:} Rafal Mantiuk and Andrew Pitts 

\section*{Introduction and Description of the Work}
Heart rate measurements from smartwatches are notoriously spurious and can, especially during exercise, provide inaccurate measurements. Optical heart rate monitors work by measuring the amount of light emitted by the monitor that is reflected by the surface of the skin. From this, the volume of blood flowing beneath the surface can be inferred, known as a plethysmogram. Tracking this value through time allows the heart rate of the user to be estimated. However, large amounts of movement, as typical during exercise, generate motion artifacts which degrade the accuracy of the measurements from smartwatches by decreasing the signal-to-noise ratio of the plethysmogram. \\ \\
Papers have shown that an accurate heart rate value can be extracted from a video recorded by a camera \cite{originalPaper}. This is because slight changes in the colour of the face, as well as slight movements of the face, allow a plethysmogram to be inferred. Any technique which can measure a plethysmogram optically via a non-contact method is known as remote photoplethysmography (rPPG) and is an active area of research. This project is concerned with the implementation of such a system which should be accessible through an Android application using a phone camera. \\ \\
Since, as some experiments have shown, the face contains a greater PPG signal than the wrist \cite{vanderKooij2019}, I would like to investigate the extent to which rPPG could be used as a replacement for smartwatches during exercise. This will involve investigating the effects of distance on accuracy as well as comparisons of different algorithms for computing the heart rate. \\ \\
It is expected that some may perform better than others in certain scenarios such as amount of movement or lighting conditions and, as a result, there is scope for combining algorithms or using heuristics for selecting appropriate implementations at runtime. In order to develop these heuristics, the effects on accuracy of numerous conditions should be examined such as, potentially, the effects of the resolution and framerate of the camera. \\ \\
This could be critical since many current rPPG systems do not run in realtime and thus, any reductions in the amount of processing required, for example, by downsampling images or not considering every frame, could help to maintain accuracy whilst improving performance. \\\\
Furthermore, recent reductions in the price of smartphones means that high-quality cameras are much more ubiquitous than heart-rate measuring equipment. An abnormal heart rate can be indicative of wider medical problems and, as a result, the ability to measure heart rates easily could be helpful for communities without regular access to healthcare. To that end, this project may be suitable for medical uses in remote areas.

\section*{Starting Point}
I have no previous experience developing Computer Vision applications and nor in Digital Signal Processing. However, I believe, Part 1A Introduction to Graphics and Part 1B Further Graphics will prove useful precursors to understanding the Computer Vision aspects of the project. Likewise, I am currently studying Part II Information Theory which I hope will help provide a grounding in some of the mathematics behind Digital Signal Processing, which in turn will help with understanding the fundamental algorithms in the field.

\section*{Work to be Done}
\subsection*{Face detection}
The face is typically the most exposed region of the body during exercise and is believed to be the easiest region to extract a reliable PPG signal from \cite{vanderKooij2019}. Given this idea, the system, from videos, will have to be able to identify faces within the videos. At present, this is likely to involve a sweep of the frame using the Viola-Jones algorithm to box any faces in the frame. This initial stage may then be followed by picking out the exact area of the face from within the box.

\subsection*{Region of interest selection}
Within the region of the face itself different areas contain differing amounts of information regarding the pulse. For example, eyes give no information about the pulse and it is speculated that some regions of the face such as the forehead and cheeks give more information than other regions like the nose. Several algorithms will be developed to locate these regions and their performance will be compared.

\subsection*{Region tracking}
Once a region is selected, it must be tracked between frames. This is because the colour of the face itself is of no real concern but the frequency at which the colour changes is the means by which we can infer the heart rate. Thus if different regions in each frame are tracked then this frequency will begin to diverge from the true value. This is likely to use some kind of Optical Flow algorithm, the exact nature of which will be investigated.

\subsection*{Signal processing of RGB signals}
The resulting signals extracted from the region of interest (ROI), will be very noisy and will require the use of signal processing techniques before applying a Fourier transform to extract the prominent frequency. It is expected that this will require the use of a Blind Source Separation technique such as Independent Component Analysis to separate the independent sources contributing to the signal.

\subsection*{Android application}
An Android application will be developed for easy testing of the system by allowing users to record videos of themselves and estimate their own heart rate.

\subsection*{Interaction between Android application and video analysis}
The provision of signal processing and computer vision libraries isn’t particularly strong in the JVM languages which are how Android applications are typically programmed. As a result, it is likely that the the video analysis software will be written as a separate program, most likely in Python or Julia, which will then interact with the Android application via pipes. 

\subsection*{Evaluation}
I hope to conduct my own experiments on the performance of the system relative to a smartwatch when compared to a known ground truth. As well as experiments across a variety of skin tones and lighting conditions.

\subsection*{Test bench application}
In order to test the developed system, in comparison with a traditional smartwatch, an application for extracting the heart rates measured by the smartwatch will have to be developed. This will take the form of a logging application which will be able to run in the background on the watch.

\section*{Possible Extensions}
\begin{itemize}
    \item Tracking multiple faces in a single video
    \item Measuring heart rate whilst exercising
    \begin{itemize}
        \item \textbf{Increased tracking}: the core program is only expected to deal with minor movements such as limited head movement, an increase in the stability of the tracking algorithms will almost certainly be required to deal with exercising users.
        \item \textbf{Dealing with increased distance from camera}: an upscaling algorithm might be required to be able to extract heart rate from a distance, since the face region of the user might be small which would lead to a decrease signal-to-noise ratio, upscaling may help to increase this.
        \item \textbf{Increasingly noisy signals}: more rigorous signal processing will be required to remove additional noise caused by movement from the signals. This is because the frequency of movement may fall within the same range as the expected frequency of the heart \cite{Peng2014}, hence simple band-pass filters will no longer work.
    \end{itemize}
    \item End to End Deep Learning Approach: Recent papers \cite{10.1007/978-3-030-01216-8_22} have shown that we can use models based on Convolutional Neural Networks taking a pair of frames to estimate the change in pulse. An implementation of this system could then be compared to the core program.
    \item Most systems outlined in the literature carry out off-line processing of the video frames, however, many provisions exist on Android for parallelized computation on images \cite{li2018differentiable}. These could be utilised to develop a real-time application.
\end{itemize}

\section*{Success Criteria}
\begin{itemize}
\item Develop an Android application that allows users to estimate their own heart rates
\item Stationary users in appropriate lighting conditions should be able to measure their heart rate with reasonable accuracy
\end{itemize}

\section*{Timetable}
I have created a timetable consisting of 12 2-week periods starting on 26/10/2019 and finishing on 25/4/2020.
\begin{enumerate}
\item \textbf{26/10/2019 - 09/11/2019} \\
I will begin by conducting research into the OpenCV library for Computer Vision and assessing the provisions already present in the library. I should also be prototyping, most probably in Python, the face detection.

\item \textbf{09/11/2019 - 23/11/2019} \\
The process of researching and prototyping should continue, with a rough structure of the analysis software in place, meaning that the program should be able to receive streams of frames and detect faces in each frame.

\item \textbf{23/11/2019 - 07/12/2019}\\
Various Region of Interest selection algorithms should be implemented and tested, although their effect on accuracy cannot be measured yet, they should be tested for correctness.

\item \textbf{21/12/2019 - 04/01/2020}\\
Several different point tracking algorithms should be selected and included in the program. At present, this is likely to include Lucas-Kanade optical flow (a sparse technique) and dense optical flow.

\item \textbf{04/01/2020 - 18/01/2020}\\
Implement signal processing pipeline to allow for cleaning of RGB signal and extracting heart rate. This will complete the analysis software and I expect tweaking of the pipeline to occur at this stage based on any accuracy issues.

\item \textbf{18/01/2020 - 01/02/2020}\\
Write the progress report and begin writing an Android application to allow for users to measure their own heart rate, should also allow for overlaying heart rate on the image.

\item \textbf{01/02/2020 - 15/02/2020}\\
Write serialisation code to allow for communication between Android application and the analysis program. With the completion of this, the core project should be finished.

\item \textbf{15/02/2020 - 29/02/2020}\\
Begin drafting the introduction chapter of the dissertation as well as beginning work on the code required to evaluate the project. This will include code for running experiments and measuring accuracy.

\item \textbf{29/02/2020 - 14/03/2020} \\
Continue writing the introduction chapter and begin work on the preparation chapter, as well as beginning work on the extensions.

\item \textbf{14/03/2020 - 28/03/2020}\\
Work on the extensions should be concluded and the implementation chapter finished. Feedback on the previous chapters should be sought out and appropriate modifications made.

\item \textbf{28/03/2020 - 11/04/2020}\\
Finish writing the dissertation and seek final supervisor comments.

\item \textbf{11/04/2020 - 25/04/2020}\\
Incorporate suggestions and hand in the final version.
\end{enumerate}

\section*{Resource Declaration}
I will use my own laptop, a Lenovo 510s with a 2.5 GHz Intel Core i5 CPU and 8GB of RAM. I accept full responsibility for this machine and I have made contingency plans to protect myself against hardware and/or software failure. All my code for the project and the dissertation itself will be pushed to a Git repository that will be hosted on GitHub. I will push to this regularly so that, in the case of an issue with my laptop, I will be able to resume work promptly. I own an Android phone (Google Pixel 3A XL) which will be used as part of the development process. I have also been provided on loan with a smartwatch by my supervisor, the Fossil Q smartwatch which has Wear OS installed, will be used for any evaluation requiring a smartwatch comparison.

% \bibliography{references} 

% \bibliographystyle{ieeetr}
% \end{document}
